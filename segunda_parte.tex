\documentclass[preprint,10pt]{article}

\usepackage{amsfonts}
\usepackage{amsmath}
\usepackage{amssymb}
\usepackage{enumerate}
\usepackage[english]{babel}
\usepackage[utf8]{inputenc}
\usepackage{bbm}
\usepackage{mathrsfs}
\usepackage{color}
\usepackage{refcheck}
\usepackage{multirow}
\usepackage{graphicx}
\usepackage{epsfig}
%\usepackage[colorlinks=true,citecolor=red,linkcolor=blue,pdfpagetransition=Blinds]{hyperref}
\usepackage{fullpage}
\usepackage{pgfplotstable}
\usepackage{lipsum}
\usepackage{cases}
\providecommand{\keywords}[1]{\noindent {\textbf{Keywords:}} #1}

%\usepackage[margin=2cm]{geometry}

\usepackage{color}
\newcommand{\R}{{\mathbb{R}}}
\newcommand{\Rn}{{\mathbb{R}}^n}
\newcommand{\N}{{\mathbb{N}}}
\newtheorem{theorem}{Theorem}
\newtheorem{acknowledgement}[theorem]{Acknowledgement}
\newtheorem{algorithm}[theorem]{Algorithm}
\newtheorem{axiom}[theorem]{Axiom}
\newtheorem{case}[theorem]{Case}
\newtheorem{claim}[theorem]{Claim}
\newtheorem{conclusion}[theorem]{Conclusion}
\newtheorem{condition}[theorem]{Condition}
\newtheorem{conjecture}[theorem]{Conjecture}
\newtheorem{corollary}[theorem]{Corollary}
\newtheorem{criterion}[theorem]{Criterion}
\newtheorem{definition}[theorem]{Definition}
\newtheorem{example}[theorem]{Example}
\newtheorem{exercise}[theorem]{Exercise}
\newtheorem{lemma}[theorem]{Lemma}
\newtheorem{notation}[theorem]{Notation}
\newtheorem{problem}[theorem]{Problem}
\newtheorem{proposition}[theorem]{Proposition}
\newtheorem{remark}[theorem]{Remark}
\newtheorem{assumption}[theorem]{Assumption}
\newtheorem{solution}[theorem]{Solution}
\newtheorem{summary}[theorem]{Summary}
\newenvironment{proof}[1][Proof]{\noindent\textbf{#1.} }{\ \rule{0.5em}{0.5em}}
\numberwithin{equation}{section} 
\numberwithin{theorem}{section}

\def\dx{\,\textnormal{d}x}
\def\dt{\textnormal{d}t}
\def\d{\,\textnormal{d}}
\def\cbd{\mathcal O}
\def\normal{n}
\def\supp{\textnormal{supp}\,}
\def\csbd{\chi_{\mathcal O}}
\newcommand\csin[1]{\mathbf1_{#1}}
\def\dx{\,\textnormal{d}x}
\def\dt{\textnormal{d}t}
\def\d{\,\textnormal{d}}

\DeclareMathOperator*{\esssup}{ess\,sup}
\newcommand\magenta[1]{{\color{magenta} #1}}


\begin{document}

\title{\bf The super paper}

\author{ V\'ictor Hern\'andez-Santamar\'ia \thanks{Institut de Math\'{e}matiques de
    Toulouse, UMR 5219,
    Universit\'e de Toulouse, CNRS, UPS IMT, F-31062 Toulouse Cedex 9,
    France. E-mail: \texttt{victor.santamaria@math.univ-toulouse.fr}} \and Liliana Peralta \thanks{E-mail: \texttt{liliana\_peralta@uaeh.edu.mx}}}


\maketitle

\abstract{
\magenta{cambiar el abstract de acuerdo a los resultados finales}
%In this paper, we present some controllability results for parabolic equations in the framework of hierarchic control. In the first part, we present a Stackelberg strategy combining the concept of controllability with robustness: the main control (the leader) is in charge of a null-controllability objective while a secondary control (the follower) solves a robust control problem, this is, we look for an optimal control in the presence of the worst disturbance. We improve previous results by considering that either the leader or follower control acts on a small part of the boundary.  
%
%On a second result, we present a Stackelberg-Nash strategy for the heat equation where we act on the system through several followers --solving a Nash multi-objective equilibrium-- and where the classical null-controllability objective for the leader is replaced by an insensitizing control problem.}


\keywords{Hierarchic control, robust control, Carleman estimates, controllability.}

\section{Introduction}
%
%\subsection{The hierarchic control problem}
%
Optimization and control problems arise in many applications of engineering and mathematics. Traditionally, such problems deal with a single objective: minimize cost, maximize benefit, etc. However, when studying more realistic and complex situations, it is desirable to include several different objectives and therefore the introduction of multi-objective optimization is essential. 

%Unlike the single-objective case, there are different strategies in order to choose the controls in a multi-objective problem and the election greatly depends on the character of the problem. Different notions for the solution of a multi-objective problem were introduced in economics and game theory, see \cite{Nash}, \cite{Pareto}, \cite{Stackelber}. %In this sense, when
%%dealing with multi-objective optimization problems, a concept of solution needs to be clarified.

In the framework of control of PDEs, the so-called hierarchic control was introduced in \cite{LionsHier,LionsSta} by J.-L. Lions to study a bi-objective control problem for the wave and heat equation, respectively. In these works, the hierarchic control method is proposed as a tool to combine the concepts of optimal control and controllability. Such methodology employs the notion of Stackelberg optimization (\cite{Stackelber}) to deal with a multi-objective decision problem where one of the participants, the \emph{leader}, is in charge of a controllability objective and the other participant, the \emph{follower}, deals with an optimal control one. 

In the recent past, several authors have applied successfully the hierarchic control method for a wide variety of equations and solving different kind of objectives, see, among others, \cite{a_araujo,araruna,araruna1,carreno,Guillen,vhs_deT_rob,jesus,montoya}. In  particular, in \cite{vhs_deT_rob}, the authors presented a methodology to combine the notion of hierarchic control introduced by Lions in \cite{LionsSta} with the concept of robust control appearing in optimal control problems (see \cite{aziz,temam,temam_nonlinear}). 

However, all of the previous works have one thing in common: they deal with hierarchic control when the controls are localized in the interior of the domain, namely, for distributed controls. As far as we know, there is only one paper dealing with the boundary case: in \cite{da_silva}, the authors study a Stackelberg-Nash strategy for  (semilinear) parabolic equations with the possibility of the leader or the followers being placed on the boundary. 

Here, adapting some arguments in \cite{da_silva}, we extend and discuss some results concerning the robust hierarchic strategy for the heat equation introduced in \cite{vhs_deT_rob} but using boundary controls instead of distributed ones. \magenta{As in other control problems, the change from distributed to boundary controls imposes additional difficulties which, together with the intricacy of the hierarchic methodology,  allows us to obtain results for some configurations and hence motivate interesting open problems. }

\subsection{The problem and its formulation}

In this paper, we are interested in a robust control strategy for the heat equation where we assume that we can act on the dynamics of the system through a hierarchy of controls. Especially, we are interested in the case where some of the controls act through a (small) portion of the boundary. To fix ideas, we begin by explaining one of the control problems addressed in this paper. 

Let us consider a bounded open set $\Omega\subset \mathbb{R}^N$, $N\geq 1$ with boundary $\partial \Omega$ of class $\mathcal C^2$. Let $\omega\subset \Omega$ be a nonempty open subset and $\cbd$ be a nonempty open subset of $\partial \Omega$.  Given $T>0$, we will use the notation $Q:=\Omega\times(0,T)$ and $\Sigma:=\partial \Omega\times(0,T)$, while $n(x)$ will denote the outward unit normal vector at the point $x\in \partial \Omega$. 

Let us consider the system
%
\begin{equation}\label{heat_lin}
\begin{cases}
y_t-\Delta y=\csin{\omega}h+\psi, & \text{in Q}, \\
y=v\csbd &\text{on } \Sigma, \\
y(x,0)=y^0(x), & \text{in } \Omega.
\end{cases}
\end{equation}
%
where $y_0\in L^2(\Omega)$ is a given initial datum and $\psi\in L^2(Q)$ is an unknown perturbation. 

In \eqref{heat_lin}, $y=y(x,t)$ is the state while $h=h(x,t)$ and $v=v(x,t)$ are control functions applied on $\omega$ and $\mathcal O$, respectively. Here $\csin{\omega}$ is the characteristic function of the set $\omega$ and $\csbd$ is a smooth nonnegative function such that $\supp\csbd=\overline \cbd$. \magenta{checar si es igualdad o $\subset\subset$}

The intuitive idea of the robust hierarchic control is to choose ``simultaneously'' the control functions  $v$ and $h$ in such way that the following optimality problems are solved: 
%
\begin{enumerate}
\item find the ``best'' control $v$ such that the solution to $y$ is ``not too far'' from a desired target $y_d$ even in the presence of the ``worst'' disturbance $\psi$, and
\item find the minimal norm control $h$ such that $y(\cdot,T)=0$. 
\end{enumerate}

Seen independently, Problem 1 (i.e. $h\equiv 0$) is a classical robust control problem (cf. \cite{temam,temam_nonlinear,aziz}) which looks for a control such that a given cost functional achieves its minimum in presence of the worst disturbance possible. Problem 2 (i.e. $v\equiv\psi\equiv 0$) is a classical null controllability problem and it has been studied for a broad class of systems described by parabolic PDEs, see for instance \cite{cara_guerrero} and the references within.

Consider a nonempty open set $\mathcal O_d\subset \Omega$ and define the cost functional
%
\begin{equation}\label{func_rob}
J_r(v,\psi;h)=\frac{1}{2}\iint_{\mathcal O_d\times(0,T)}|y-y_d|^2\dx\dt+\frac{1}{2}\left[\ell^2\iint_{\partial \Omega\times(0,T)}|v|^2\d\sigma\dt-\gamma^2\iint_Q|\psi|^2\dx\dt\right]
\end{equation}
%
%where $\ell,\gamma>0$ are constants and $y_d\in L^2(\mathcal O_d\times(0,T))$ is given. This functional is used to formulate and solve Problem 1. Indeed, we will look for a saddle point $(\bar v,\bar \psi)$ which simultaneously maximize $J_r$ with respect to $\psi$ and minimize it with respect to $v$, while maintaining the state $y$ ``close enough'' to a desired target $y_d$ in the observation domain $\mathcal O_d\times(0,T)$. The parameters $\ell$ and $\gamma$ will play a key role and take into account the relative weight of each term. 
%
where $\ell,\gamma>0$ are constants and $y_d\in L^2(\mathcal O_d\times(0,T))$ is given. This functional is used to formulate and solve Problem 1. Indeed, we will look for a saddle point $(\bar v,\bar \psi)$ which simultaneously maximize $J_r$ with respect to $\psi$ and minimize it with respect to $v$. The parameters $\ell$ and $\gamma$ play a key role and take into account the relative weight of each term: the term $-\gamma^2\|\psi\|^2_{L^2(Q)}$ constrains the magnitude of the disturbance allowed in the optimization process while the term associated to $\ell^2\|v\|^2_{L^2(\partial\Omega\times(0,T))}$ moderates the effort made by the control. 

Now, we are in position to describe the Robust hierarchic control problem. According to the formulation introduced by Lions \cite{LionsSta} (originally from \cite{Stackelber}), we denote $h$ as the leader control and $v$ as the follower control. Then, the proposed methodology consints of two parts:

\begin{enumerate}
\item[(i)] For a fixed leader $h\in L^2(\omega\times(0,T))$, we look for an optimal pair $(\bar v,\bar \psi)$ solving the robust control problem:
%
\begin{definition}\label{defi_rob}
Let $h\in L^2(\omega\times(0,T))$ be fixed. The control $\bar v\in L^2(\partial \Omega\times(0,T))$, the disturbance $\bar \psi\in L^2(Q)$ and the state $\bar y=\bar y(h,\bar v,\bar \psi)$ solution to \eqref{heat_lin} associated to $(\bar v,\bar \psi)$ are said to solve the robust control problem when a saddle point $(\bar v,\bar \psi)$ (which depends on $h$) of the cost functional $J_r$ is reached, namely
%
\begin{equation}\label{saddle}
J_r(\bar v,\psi;h)\leq J_r(\bar v,\bar \psi;h)\leq J_r(v,\bar \psi;h), \quad \forall (v,\psi)\in L^2(\partial \Omega\times(0,T))\times L^2(Q).
\end{equation}
%
In this case,
%
\begin{equation}\label{saddle_minmax}
J_r(\bar v,\bar \psi;h)=\max_{\psi\in L^2(Q)}\min_{v\in L^2(\partial\Omega\times(0,T))}J_r(v,\psi;h)=\min_{v\in L^2(\partial\Omega\times(0,T))}\max_{\psi\in L^2(Q)}J_r(v,\psi;h).
\end{equation}
%
\end{definition} 
%
Under certain conditions, we will prove that there exists a unique pair $(\bar v,\bar \psi)$ and the associated state $\bar y=\bar y(h,\bar v,\bar\psi)$ such that \eqref{saddle} holds.
%
\item[(ii)] After identifying the saddle point for each $h$, we look for the control of minimal norm $\bar h$ satisfying null controllability constraints, i.e., we look for an optimal control $\bar h$  such that
%
\begin{equation}\label{opt_leader}
J(\bar h)=\min_{h\in L^2(\omega\times(0,T))}\frac{1}{2}\iint_{\omega\times(0,T)}|h|^2\dx\dt, \quad \text{subject to } y(\cdot,T;\bar v,\bar \psi)=0.
\end{equation}
%%
%subject to 
%%
%\begin{equation}\label{opt_rest}
%\bar y(\cdot,T;h,\bar v(h),\bar \psi(h))=0 \quad \text{in}\quad  H^{-1}(\Omega).
%\end{equation}
\end{enumerate}

\begin{remark}
As in \cite{LionsSta} and other related papers, we address the multi-objective optimization problem by solving the mono-objective problems \eqref{saddle_minmax} and \eqref{opt_leader}. Note, however, that in the second minimization problem the solution of the robust control is fixed and therefore its characterization needs to be considered.
\end{remark}

\subsection{Main results}

%

In a first result, we address the robust hierarchic control of system \eqref{heat_lin}, that is, the case where the follower control is applied on the boundary and the leader control is a distributed one. In this regard, we have the following
%
\begin{theorem}\label{teo_main1}
Assume that $\omega\cap\mathcal O_d\neq \emptyset$. Then, there exist $\gamma_0$, $\ell_0$ and a positive function $\rho=\rho(t)$ blowing up at $t=T$ such that for any $\gamma>\gamma_0$, $\ell>\ell_0$, $y^0\in L^2(\Omega)$ and $y_d\in L^2(\mathcal O_d\times(0,T))$ verifying 
%
\begin{equation}\label{integ_yd}
\iint_{\mathcal O_d\times(0,T)}\rho^2|y_d|^2dxdt<+\infty,
\end{equation}
%
we can find a leader control $h\in L^2(\omega\times(0,T))$ and a unique saddle point $(\bar v,\bar\psi)\in L^2(\partial\Omega\times(0,T))\times L^2(Q)$, for the functional given by \eqref{func_rob}, such that the associated solution to \eqref{heat_lin} verifies $y(\cdot,T)=0$ in $\Omega$. 
%
\end{theorem}

As is common in robust control problems, the assumption on $\gamma$ means that the possible disturbances spoiling the control objectives should have moderate $L^2$-norms. Indeed, without this condition, it is not possible to prove the existence of the saddle point. On the other hand, the assumption on the target $y_d$ means that should approach 0 as $t\to T$. This is a standard feature in some null controllability problems and has been discussed, for instance, in \cite{araruna,deteresa2000}. 

We shall prove Theorem \ref{teo_main1} in two steps. In a first one, we will adapt the methodology in \cite{vhs_deT_rob} to the boundary case  to prove the existence and uniqueness of a saddle point to \eqref{func_rob}. As common in optimal control problems, the solution can be characterized by means of an optimality system leading to a coupled system. In the second part, we will use Carleman estimates for parabolic equations with non-homogenous boundary terms to deduce an observability inequality for a suitable adjoint system, which will imply the desired null controllability objective. 

In this paper, we are also interested in studying different configurations for the positioning of the boundary controls. A first question that arises naturally is the possibility to exchange the position of the leader $h$ and the follower control $v$. Adapting some of the arguments in \cite{da_silva}, we will see that this is in fact possible by considering systems of the form 
%
\begin{equation}\label{heat_dif}
\begin{cases}
z_t-\Delta z=\csin{\mathcal B_1}v+\csin{\mathcal B_2}\psi, & \text{in Q}, \\
z=h\csbd &\text{on } \Sigma, \\
z(x,0)=z^0(x), & \text{in } \Omega.
\end{cases}
\end{equation}
%
where \magenta{$\mathcal B_i \subsetneq \Omega$}, $i=1,2,$ are nonempty open subsets and $z^0(x)\in H_0^1(\Omega)$ is a given initial datum. 

The same methodology presented above can be used to address the Robust hierarchic control of \eqref{heat_dif}. In this case, the cost functional \eqref{func_rob} should be replaced by 
%
\begin{equation}\label{func_rob_dif}
K_r(v,\psi;h)=\frac{1}{2}\iint_{\mathcal O_d\times(0,T)}|y-y_d|^2\dx\dt+\frac{1}{2}\left[\ell^2\iint_{\mathcal B_1\times(0,T)}|v|^2\dx\dt-\gamma^2\iint_{\mathcal B_2\times(0,T)}|\psi|^2\dx\dt\right]
\end{equation}
%
As before, we will see that the robust control can be solved by selecting appropriate parameters $\ell$ and $\gamma$. Notice that unlike \eqref{heat_dif}, here we maximize for disturbances $\psi$ localized in the region $\mathcal B_2$. This comes from a technical reason concerning the resolution of the null controllability objective (see Section \ref{} for details), but which is not necessary to solve the robust control part. 

Once a characterization for the saddle point of $K_r$ is known, the resulting optimality system is a coupled system of PDEs. It is well-known that controllability problems using boundary controls is a difficult task for systems of two or more equations (see, e.g, \cite{assia_survey,assia_luz_new}). Here, using that the parameters $\ell,\gamma$ coming from the solution of the robust control part are sufficiently large, we will combine Carleman estimates with boundary observations together with weighted energy estimates to obtain an observability inequality for a system with two equations and only one observation at the boundary.

The result can be summarized as follows. 
%
\begin{theorem}\label{teo_main2}
Assume that 
%
\begin{equation}\label{loc_teo2}
\mathcal O_d\cap \mathcal B_i =\emptyset \quad\text{and}\quad \bar{\mathcal O}\subset \partial \mathcal B_i, \quad i=1,2.
\end{equation}
%
Then, there exist $\gamma_0$, $\ell_0$ and a positive function $\rho=\rho(t)$ blowing up at $t=T$ such that for any $\gamma>\gamma_0$, $\ell>\ell_0$, $y^0\in H_0^1(\Omega)$ and $y_d\in L^2(\mathcal O_d\times(0,T))$ verifying 
%
\begin{equation}%\label{integ_yd}
\iint_{\mathcal O_d\times(0,T)}\rho^2|y_d|^2dxdt<+\infty,
\end{equation}
%
we can find a leader control $h\in L^2(\omega\times(0,T))$ and a unique saddle point $(\bar v,\bar\psi)\in L^2(\mathcal B_1\times(0,T))\times L^2(\mathcal B_2\times(0,T))$, for the functional given by \eqref{func_rob_dif}, such that the associated solution to \eqref{heat_dif} verifies $y(\cdot,T)=0$ in $\Omega$. 
%
\end{theorem}

Hypothesis \eqref{loc_teo2} plays a fundamental role in the selection of the weight functions participating in the Carleman estimates needed to prove Theorem \ref{teo_main2}. This is not a common selection and the proof can only be achieved thanks to the special structure of the adjoint system (see eq. \eqref{}). This special selection has been recently used in other hierarchic control problems, see \cite{da_silva}. 

\magenta{Lo que viene abajo aun no es definitivo, pero creo que se puede probar y complementaria el trabajo de buena forma. EL chiste es mencionar que es interesante ver que se puede hacer en el caso donde todos los controles estan sobre la frontera. La cosa se pone fea y la idea es usar una idea que hicimos en nuestro corrigendum. Tengo casi todo para usar esa idea pero hay un estimado que no se si sea cierto. De cualquier forma no es un estimado comun que venga en los libros pero creo qeu Umberto (tal vez franck, pero no se) puede saber.}

A second question that arises in this context is if it is possible to put both controls on the boundary of the system, that is
%
\begin{equation*}%\label{heat_all}
\begin{cases}
w_t-\Delta w=\psi, & \text{in Q}, \\
w=h\chi_{\mathcal O_1}+ v\chi_{\mathcal O_2}&\text{on } \Sigma, \\
w(x,0)=w^0(x), & \text{in } \Omega.
\end{cases}
\end{equation*}
%
We provide a partial answer to this problem for the case where $\psi\equiv 0$ and the cost functional corresponding to the optimal control problem takes the form
%
\begin{equation}
I(v;h)=\frac{1}{2}\iint_{\mathcal O_d\times(0,T)}|w-w_d|^2\dx\dt+\frac{\ell^2}{2}\iint_{\partial\Omega\times(0,T)}\rho_\star^2|v|^2\d\sigma\dt
\end{equation}
%
where $\rho_\star=\rho_\star(t)$ is a suitable positive weight function blowing up at $t=T$. This is a classical o



\vspace{5 cm}
\%\%\%\%\%\%\%\%\%\%\%\%\%\%\%\%\%\%\%\%\%\%\%\%\%\%\%\%\%\%\%\%\%\%\%\%\%\%\%\%\%\%\%\%\%\%\%\%\%\%\%

In \cite{vhs_deT_rob}, the authors presented a methodology to solve simultaneously the above problems. They propose to combine the notion of hierarchic control introduced by Lions in \cite{LionsSta} with the concept of robust control appearing in optimal control problems (see \cite{aziz,temam,temam_nonlinear}). The hierarchic control technique uses the concept of Stackelberg optimization and has been recently used to study several multi-objective control problems, see for instance \cite{araruna,AMR, Glowinski,Guillen,Limaco,LionsHier}. 

In this paper, we use the methodology introduced \cite{vhs_deT_rob} to study the controllability of systems \eqref{heat_b1} and \eqref{heat_b2}. Below, we introduce the control problems and describe the method to solve them. Unlike \cite{vhs_deT_rob}, where the control functions $v$ and $h$ are localized in the interior of the domain, here we study controllability properties when either of the controls involved is applied on the boundary of the system. This change introduces additional difficulties.  

\subsection{The problem and its formulation}
%
We begin by describing the control problem for system \eqref{heat_b1}. As mentioned before, it consists of two parts: a robust control problem and a controllability one.



In the second problem, we look for a control $h\in L^2(\omega\times(0,T))$ minimizing 
%
\begin{equation}\label{func_null}
J(h)=\frac{1}{2}\iint_{\omega\times(0,T)}|h|^2dxdt \quad\text{subject to}\quad y(\cdot,T)=0. 
\end{equation}
%
This is a classical null controllability problem. In the case where $\psi=v=0$, it its well-known that system \eqref{heat_b1} is null controllable, see, for instance, \cite{fursi,cara_guerrero}. 

It is well-known that for every $h\in L^2(\omega\times(0,T))$, $v\in L^2(\Gamma\times(0,T))$, $\psi\in L^2(Q)$, and $y_0\in H^{-1}(\Omega)$, system \eqref{heat_b1} admits a unique weak solution (defined by transposition) that satisfies
%
\begin{equation}
y\in L^2(Q)\cap C^0([0,T];H^{-1}(\Omega)).
\end{equation} 

Following the notation of \cite{LionsSta} (originally introduced in \cite{Stackelber}), we denote $h$ as the \textit{leader} control and $v$ as the \textit{follower} control. As proposed in \cite{vhs_deT_rob}, we will use the following hierarchic control strategy to solve the optimization problems related to the cost functionals \eqref{func_rob}--\eqref{func_null}: 
%
\begin{enumerate}
\item 
%
Under certain conditions, we will prove that there exists a unique pair $(\bar v,\bar \psi)$ and the associated state $\bar y$ such that \eqref{saddle} holds.
%
\item After identifying the saddle point for each leader control $h$, we look for an optimal control $\hat h$ such that 
%
\begin{equation}\label{opt_leader}
J(\hat h)=\min_{h\in L^2(\omega\times(0,T))}J(h), 
\end{equation}
%
subject to 
%
\begin{equation}\label{opt_rest}
\bar y(\cdot,T;h,\bar v(h),\bar \psi(h))=0 \quad \text{in}\quad  H^{-1}(\Omega).
\end{equation}
%
\end{enumerate}

\begin{remark}
As in \cite{LionsSta} and other related papers, we address the multi-objective optimization problem by solving the mono-objective problems \eqref{saddle_minmax} and \eqref{opt_leader}--\eqref{opt_rest}. Note, however, that in the second minimization problem the optimal strategy for the follower is fixed and therefore its characterization needs to be considered.
\end{remark}

Analogously, we may define the functionals for the multi-objective problem related to system \eqref{heat_b2}. We define the Hilbert spaces $H^{r,s}(\Sigma):=L^2(0,T;H^r(\Gamma))\cap H^s(0,T:L^2(\Gamma))$ and introduce the following space
%
\begin{equation}
V:=\left\{u\in H^2(\Omega): \  u=0 \ \text{on } \Gamma\backslash \mathcal S\right\}.
\end{equation}

Consider the cost functional 
%
\begin{equation}\label{func_S_rob}
K_r(v,\psi;h)=\frac{1}{2}\iint_{\Gamma_d\times(0,T)}\left|\frac{\partial z}{\partial \nu}-\zeta_d\right|^2d\sigma dt+\frac{1}{2}\left[\ell^2\iint_{\mathcal B_1\times(0,T)}|v|^2dxdt-\gamma^2\iint_{\mathcal B_2\times(0,T)}|\psi|^2dxdt\right]
\end{equation}
%
where $\Gamma_d$ is a nonempty closed subset of $\Gamma$, $\zeta_d$ is a given function and $\ell,\gamma$ are positive constants. This functional describes the robust control problem for \eqref{heat_b2}. The leader functional is the following
%
\begin{equation}
K(h):=\frac{1}{2}\|h\|^2_{H^{\frac{3}{2},\frac{3}{4}}(\mathcal S\times(0,T))}
\end{equation}
%

As before, for each leader control $h$, we will find a saddle point for the cost functional \eqref{func_S_rob}, more precisely, 
we look for an optimal pair $(\bar v,\bar \psi)$ and the associated state $\bar z=\bar z(h,\bar v,\bar \psi, z_0)$ such that 
%
\begin{equation}\label{saddle_2}
J_r(\bar v,\psi;h)\leq J_r(\bar v,\bar \psi;h)\leq J_r(v,\bar \psi;h), \quad \forall (v,\psi)\in L^2(\mathcal B_1\times(0,T))\times L^2(\mathcal B_2\times(0,T)).
\end{equation}
%
Once we prove the existence and uniqueness of the saddle point, we look for a control $\hat h\in H^{\frac{3}{2},\frac{3}{4}}(\mathcal S\times(0,T))$ verifying 
%
\begin{equation}
K(\hat h)=\min_{h} K(h)
\end{equation}
%
subject to the null controllability condition
%
\begin{equation}
\bar z(\cdot,T; h,\bar v(h), \bar \psi(h))=0 \quad \text{in } \Omega.
\end{equation}
%

\section{The case with boundary follower and distributed leader}

\subsection{Existence, uniqueness and characterization of the saddle point}

The existence of a solution $(\bar v,\bar \psi)$ to the robust control problem is based on the following result
%
\begin{proposition}\label{prop_saddle}
Let $J$ be a functional defined on $X\times Y$, where $X$ and $Y$ are non-empty, closed, unbounded, convex sets. If $J$ satisfies
%
\begin{enumerate}
\item $\forall \psi\in Y$, $v\mapsto J(v,\psi)$ is convex lower semicontinuous,
\item $\forall v\in X$, $\psi\mapsto J(v,\psi)$ is concave upper semicontinuous,
\item $\exists \psi_0\in X$ such that $\lim_{\|v\|_{X}\to+\infty}J(v,\psi_0)=+\infty$, 
\item $\exists v_0\in Y$ such that $\lim_{\|\psi\|_Y\to+\infty}J(v_0,\psi)=-\infty$,
\end{enumerate}
%
then the functional $J$ has at least one saddle point $(\bar v,\bar \psi)$ and
%
\begin{equation}
J(\bar v,\bar \psi)=\min_{v\in X}\sup_{\psi\in Y} J(v,\psi)=\max_{\psi\in Y}\min_{v\in X}J(v,\psi).
\end{equation}
%
\end{proposition}

 The proof can be found on \cite[Prop. 2.2, p. 173]{Ekeland}. The goal here is to apply Proposition \ref{prop_saddle} to functional \eqref{func_rob} with $X=L^2(\Gamma\times(0,T))$ and $Y=L^2(Q)$. To verify conditions 1--4 for our problem, we need the following auxiliary lemma. Recall that in the first part, the leader control $h$ is fixed.
%
\begin{lemma}\label{lemma_prop_sol1}
Let $h\in L^2(\omega\times(0,T))$ and $y_0\in L^2(\Omega)$ be given. The mapping $(v,\psi)\mapsto y(v,\psi)$ from $L^2(\Gamma\times(0,T))\times L^2(Q)$ into $L^2(Q)$ is affine, continuous, and has G\^{a}teau derivative $y^\prime(v^\prime,\psi^\prime)$ in every direction $(v^\prime,\psi^\prime)\in L^2(\Gamma\times(0,T))\times L^2(Q)$. Moreover, the derivative $y^\prime(v^\prime,\psi^\prime)$ solves the system
%
\begin{equation}\label{deriv_sys1}
\begin{cases}
y^\prime_t-\Delta y^\prime=\psi^\prime \quad &\textnormal{in $Q$}, \\
y^\prime=v^\prime\mathbf{1}_{\mathcal O} \quad &\textnormal{on $\Sigma$,} \\
y^\prime(x,0)=0 \quad &\textnormal{in $\Omega$},
\end{cases}
\end{equation}
%
\end{lemma}

\begin{proof}
The fact that the mapping $(v,\psi)\mapsto y(v,\psi)$ follows from the linearity of system \eqref{heat_b1} and Lemma \ref{lemma_trans}. The existence of the G\^{a}teau derivative and its characterization can be easily obtained by taking the limit $\lambda\to 0$ in the expression $y^\lambda:=\frac{y(v+\lambda v^\prime,\psi+\lambda\psi^\prime)-y(v,\psi)}{\lambda}.
$
\end{proof}

With this lemma, we can verify conditions 1-4 of Proposition \ref{prop_saddle}. 
%
\begin{proposition}\label{verif_cond}
Let $h\in L^2(\omega\times(0,T))$ and $y_0\in L^2(\Omega)$ be given. There exists $\gamma$ large enough such that we have 
%
\begin{enumerate}
\item $\forall \psi\in L^2(Q)$, $v\mapsto J_r(v,\psi)$ is strictly convex, lower semicontinuous,
\item $\forall v\in L^2(\Gamma\times(0,T))$, $\psi\mapsto J_r(v,\psi)$ is strictly concave upper semicontinuous,
\item $\lim_{\|v\|_{L^2(\Gamma\times(0,T))}\to+\infty}J_r(v,0)=+\infty$, 
\item $\lim_{\|\psi\|_{L^2(Q)}\to+\infty}J_r(0,\psi)=-\infty$.
\end{enumerate}
%
\end{proposition}
%
\begin{proof}
\textit{Condition 1.} By Lemma \ref{lemma_prop_sol1}, the map $v\mapsto J(v,\psi)$ is lower semicontinuous. Since $v\mapsto y(v,\psi)$ is linear and affine a straightforward computation yields the strict convexity of $J(v,\psi)$. 

\textit{Condition 2.} Again, by Lemma \ref{lemma_prop_sol1}, we deduce that the map $\psi\mapsto J(v,\psi)$ is upper semicontinuous. To prove strict concavity, we will argue as in \cite{temam_nonlinear}. Consider
%
\begin{equation}
\mathcal{G}(\tau)=J_r(v,\psi+\tau \psi^\prime).
\end{equation}
%
We will show that $\mathcal G(\tau)$ is sctrictly concave with respect to $\tau$, i.e. $\mathcal G^{\prime\prime}(\tau)<0$. We compute
%
\begin{equation}
\mathcal G^\prime(\tau)=\iint_{\mathcal O_d\times(0,T)}(y+\tau y^\prime-y_d)y^\prime-\gamma^2\iint_{Q}(\psi+\tau\psi^\prime)\psi^\prime
\end{equation}
%
where $y^\prime$ is solution to \eqref{deriv_sys1} with $v^\prime=0$. Since $y^\prime$ is clearly independent of $\tau$, a further computations yields
%
\begin{equation}
\mathcal G^{\prime\prime}(\tau)=\iint_{\mathcal O_d\times(0,T)}|y^\prime|^2-\gamma^2\iint_{Q}|\psi^\prime|^2
\end{equation}
%
Using standard energy estimates for the heat equation, it is not difficult to see that 
%
\begin{equation}
G^{\prime\prime}(\tau)\leq -(\gamma^2-C)\|\gamma^\prime\|^2_{L^2(Q)}, \quad \forall \psi^\prime\in L^2(Q)
\end{equation}
%
where $C=C(\Omega,\mathcal O_d,T)$ is a positive constant. We deduce that for a sufficiently large value of $\gamma$, we have $\mathcal G^{\prime\prime}(\tau)<0$, $\forall \tau\in \mathbb{R}$. Thus, the function $\mathcal G$ is striclty concave, and the strict concavity of $\psi$ follows immediately. 
%

\textit{Condition 3.} It is straightforward, taking $\psi=0$ in \eqref{func_rob} the result follows. 

\textit{Condition 4.} Since \eqref{heat_b1} is linear and using the estimate \eqref{est_trans}, we have
%
\begin{equation}
J(\psi,0)\leq \tilde C+C\|\psi\|_{L^2(Q)}^2-\frac{\gamma^2}{2}\|\psi\|_{L^2(Q)}^2
\end{equation}
%
where $\tilde C$ is a positive constant only depending on $y_0$, $h$ and $y_d$. Hence, for $\gamma$ large enough, condition 4 holds.
\end{proof}

Combining the statements of Propositions \ref{prop_saddle} and \ref{verif_cond}, we are able to deduce the existence of at most one saddle point $(\bar v,\bar \psi)\in L^2(\Gamma\times(0,T))\times L^2(Q)$ for the cost functional \eqref{func_rob}. In particular, the existence of this saddle point implies that 
%
\begin{equation}
\frac{\partial J_r}{\partial v}(\bar v,\bar \psi)=0 \quad \text{and}\quad \frac{\partial J_r}{\partial \psi}(\bar v,\bar \psi)=0.
\end{equation}
%
By differentiating \eqref{func_rob}, we obtain the expressions 
%
\begin{align}\label{deriv_J1}
&\left(\frac{\partial J_r}{\partial v}(v,\psi),(v^\prime,0)\right)=\iint_{\mathcal O_d\times(0,T)}(y-y_d)y_v\,dxdt+\ell^2\iint_{\Gamma\times(0,T)}vv^\prime\,d\sigma dt \\
&\left(\frac{\partial J_r}{\partial v}(v,\psi),(0,\psi^\prime)\right)=\iint_{\mathcal O_d\times(0,T)}(y-y_d)y_\psi\,dxdt-\gamma^2\iint_{Q}\psi\psi^\prime\,dxdt
\end{align}
%
where we have denoted $y_v$ and $y_\psi$ the directional derivatives of $y$ solution to \eqref{heat_b1} (see Lemma \ref{lemma_prop_sol1}) in the directions $(v^\prime,0)$ and $(0,\psi^\prime)$, respectively. 

To characterize the solution to the robust control problem, we introduce the adjoint state to system \eqref{deriv_sys1}
%
\begin{equation}\label{adj_frontera}
\begin{cases}
-q_t-\Delta q=(y-y_d)\mathbf{1}_{\mathcal O_d} \quad& \text{in }Q, \\
q=0 \quad &\textnormal{on $\Sigma$,} \\
q(x,T)=0 \quad &\textnormal{in $\Omega$}.
\end{cases}
\end{equation}

Multiplying \eqref{adj_frontera} by $y_v$ in $L^2(Q)$ and integrating by parts, we obtain 
%
\begin{equation*}
\iint_{\mathcal O_d\times(0,T)}(y-y_d)y_v\,dxdt+\iint_{\mathcal O\times(0,T)}v^\prime\frac{\partial q}{\partial \nu}\,d\sigma dt=0,
\end{equation*}
%
and upon substitution in \eqref{deriv_J1}, we get 
%
\begin{equation}
\frac{\partial J_r}{\partial v}(v,\psi)=\left.\left(\frac{\partial q}{\partial \nu}-\ell^2v\right)\right|_{\mathcal O}. 
\end{equation}
%
In a similar fashion, we multiply \eqref{adj_frontera} by $y_\psi$ in $L^2(Q)$ and integrate by parts. We deduce that 
%
\begin{equation}
\frac{\partial J_r}{\partial \psi}(v,\psi)=q-\gamma^2\psi. 
\end{equation}

Thus, so far, we have proved the following
%
\begin{proposition}
Let $y_0\in L^2(\Omega)$ and $h\in L^2(\omega\times(0,T))$ be given. Then
%
\begin{equation}
\bar v=\left.\frac{1}{\ell^2}\frac{\partial q}{\partial \nu}\right|_{\mathcal O} \quad\text{and}\quad \bar \psi=\frac{1}{\gamma^2}q
\end{equation}
% 
are the solution to the robust control problem stated in Definition \ref{defi_rob}, where $q$ is found from the solution $(y,q)$ to the coupled system 
%
\begin{equation}\label{sys_foll}
\begin{cases}
y_t-\Delta y= h\mathbf{1}_\omega+\frac{1}{\gamma^2}q \quad&\textnormal{in }Q, \\
-q_t-\Delta q=(y-y_d)\mathbf{1}_{\mathcal O_d} \quad& \textnormal{in }Q, \\
y=\frac{1}{\ell^2}\frac{\partial q}{\partial \nu}\mathbf {1}_\mathcal O, \quad q=0 \quad &\textnormal{on $\Sigma$,} \\
y(x,0)=y_0(x), \quad q(x,T)=0 \quad &\textnormal{in $\Omega$}.
\end{cases}
\end{equation}
%
which admits a unique solution for $\gamma>0$ large enough. 
%
\end{proposition}
%
\begin{remark}
As in other robust control problems (see \cite{temam_nonlinear,aziz,vhs_deT_rob}), the characterization of the saddle point leads to a system of coupled equations. This characterization needs to be considered in the following step of the hierarchic control methodology, where a null control $h$ of minimal norm must be designed.
\end{remark}
%
%
\subsection{Null controllability}
%
In this section, we will proof an observability inequality that allows to establish the null controllability of system \eqref{sys_foll}. It is classical by now that null controllability is related to the observability of a proper adjoint system (see, for instance, \cite{cara_guerrero,zab}). 

For our particular case, let us consider the adjoint of system \eqref{sys_foll}:
%
\begin{equation}\label{adj_sys_foll}
\begin{cases}
-\varphi_t-\Delta \varphi= \theta\mathbf{1}_{\mathcal O_d} \quad&\textnormal{in }Q, \\
\theta_t-\Delta \theta=\frac{1}{\gamma^2}\varphi \quad& \textnormal{in }Q, \\
\varphi=0, \quad \theta=\frac{1}{\ell^2}\frac{\partial \varphi}{\partial \nu}\mathbf {1}_\mathcal O \quad &\textnormal{on $\Sigma$,} \\
\varphi(x,T)=\varphi^T(x), \quad \theta(x,0)=0 \quad &\textnormal{in $\Omega$}.
\end{cases}
\end{equation}
%
where $\varphi^T\in H^{1}_0(\Omega)$. Then, the controllability of \eqref{sys_foll} can be characterized in terms of appropriate properties of the solutions to \eqref{adj_sys_foll}. 
More precisely, we have

\begin{proposition}\label{prop_control}
Assume that $\omega\cap\mathcal O_d\neq \emptyset$. The following properties are equivalent
%
\begin{enumerate}
\item There exists a positive constant $C$, such that for any $y_0\in H^{-1}(\Omega)$ and any $y_d\in L^2(Q)$ such that \eqref{integ_yd} holds, there exists a control $h\in L^2(\omega\times(0,T))$ of minimal norm such that 
%
\begin{equation}
\|h\|_{L^2(\omega\times(0,T))}\leq \sqrt C\left(\|y_0\|_{H^{-1}(\Omega)}+\|\rho y_d\|_{L^2(Q)}\right)
\end{equation}
%
and the associated state satisfies $y(T)=0$, where $y$ is the first component of \eqref{sys_foll}.  
%
\item There exist a positive constant $C$ and a weight function $\rho$ blowing up at $t=T$ such that the observability inequality 
%
\begin{equation}\label{obs_ineq_1}
\|\varphi(0)\|_{H^1_{0}(\Omega)}^2+\iint_Q \rho^2|\theta|^2dxdt\leq C\iint_{\omega\times(0,T)}|\varphi|^2dxdt
\end{equation}
%
holds for every $\varphi^T\in H^1_{0}(\Omega)$, where $(\varphi,\theta)$ is the solution to \eqref{adj_sys_foll} associated to the initial datum $\varphi^T$. 
\end{enumerate}
%
\end{proposition}

The equivalence between the statements of Proposition \ref{prop_control} is by now standard and relies on several classical arguments, so we omit the proof. For the interested reader, we refer to \cite{araruna,vhs_deT_rob} where null controllability properties for similar hierarchic control problems are addressed. Our task now is to prove the observability inequality \eqref{obs_ineq_1}. It will be consequence of global Carleman inequalities and some energy estimates

\subsubsection*{The Carleman estimate}

We introduce several weight functions that will be useful in the remainder of this paper. Since \eqref{cond_omega} holds, there exists a nonempty open set $\omega_0\subset\subset\omega\cap\mathcal O_d$. Let $\eta^0\in C^2(\overline\Omega)$ be a function verifying
%
\begin{equation}
\eta^0(x)>0 \ \text{in} \ \Omega, \quad \eta^0=0 \ \text{on} \ \Gamma, \quad |\nabla\eta^0|>0 \ \text{in} \ \overline{\Omega}\backslash\omega_0.
\end{equation}
%
The existence of such a function is given in \cite{fursi}. Let $l\in C^\infty([0,T])$ be a positive function in $(0,T)$ satisfying 
%
\begin{equation}
\begin{split}
&l(t)=t \quad \text{if } t\in [0,T/4], \quad l(t)=T-t \quad \text{if } t\in [3T/4,T], \\
&l(t)\leq l(T/2), \quad \text{for all } t\in[0,T]. 
\end{split}
\end{equation}
%
Then, for all $\lambda\geq 1$ we consider the following weight functions:
%
\begin{equation}\label{weights_l} 
\begin{split}
&\alpha(x,t)= \frac{e^{2\lambda\|\eta^0\|_\infty}-e^{\lambda\eta^0(x)}}{l^m(t)}, \quad \xi(x,t)=\frac{e^{\lambda \eta^0(x)}}{l^m(t)} \\
&\alpha^*(t)=\max_{x\in\overline{\Omega}}\alpha(x,t), \quad \xi^*(t)=\min_{x\in\overline\Omega} \xi(x,t).
\end{split}
\end{equation}
%
The following notation will be used to abridge the estimates
%
\begin{equation*} 
\begin{gathered}
I(s;u):=s^{-1}\iint_{Q}e^{-2s\alpha}\xi^{-1}(|u_t|^2+|\Delta u|^2)dxdt+s\iint_Qe^{-2s\alpha}\xi|\nabla u|^2dxdt+s^3\iint_Qe^{-2s\alpha}\xi^3|u|^2dxdt \\
\widetilde I(s;u):=s^{-1}\iint_{Q}e^{-2s\alpha}\xi^{-1}|\nabla u|^2dxdt+s\iint_{Q}e^{-2s\alpha}\xi |u|^2dxdt 
\end{gathered}
\end{equation*}
%
for some parameter $s>0$. 

We state a Carleman estimate, due to \cite{ima_yama_puel}, which holds for the solutions of heat equations witn non-homogeneous Dirichlet boundary conditions:
%
\begin{lemma}
Let us assume $u_0\in L^2(\Omega)$, $f\in L^2(Q)$ and $g\in H^{\frac{1}{2},\frac{1}{4}}(\Sigma)$. Then, there exists a a constant $\lambda_0$, such that for any $\lambda\geq \lambda_0$ there exist constants $C>0$ independent of $s$ and $s_0(\lambda)>0$, such that the solution $y\in W(Q)$ of 
%
\begin{equation}
\begin{cases}
u_t-\Delta u=f \quad& \textnormal{in } Q, \\
u=g \quad& \textnormal{on } \Sigma, \\
u(0)=u_0 \quad& \textnormal{in } \Omega. 
\end{cases}
\end{equation}
%
satisfies
%
\begin{equation}\label{car_boundary}
\begin{split}
\widetilde I(s;u)\leq C&\left(s^{-\frac{1}{2}}\|e^{-s\alpha}\xi^{-\frac{1}{4}}g \|_{H^{\frac{1}{2},\frac{1}{4}}(\Sigma)}^2+s^{-\frac{1}{2}}\|e^{-s\alpha}\xi^{-\frac{1}{4}+\frac{1}{m}}g\|^2_{L^2(\Sigma)}\phantom{\iint_{\omega\times(0,T)}}\right. \\
&\left.\quad+s^{-2}\iint_Qe^{-2s\alpha}\xi^{-2}|f|^2dxdt+s\iint_{\omega_0\times(0,T)}e^{-2s\alpha}\xi|u|^2dxdt\right)
\end{split}
\end{equation}
%
for every $s\geq s_0(\lambda)$. 
\end{lemma}

The second result we need is the classical Carleman estimate for the linear heat equation (see, for instance, \cite{fursi,cara_guerrero}):
\begin{lemma}
Let us assume $u^T\in L^2(\Omega)$ and $f\in L^2(Q)$. Then, there exists a a constant $\lambda_1$, such that for any $\lambda\geq \lambda_1$ there exist constants $C>0$ independent of $s$ and $s_1(\lambda)>0$, such that the solution of 
%
\begin{equation}
\begin{cases}
u_t+\Delta u=f \quad& \textnormal{in } Q, \\
u=0 \quad& \textnormal{on } \Sigma, \\
u(T)=u^T(x) \quad& \textnormal{in } \Omega. 
\end{cases}
\end{equation}
%
satisfies
%
\begin{equation}\label{car_clasica}
\begin{split}
I(s;u)\leq C&\left(\iint_Qe^{-2s\alpha}|f|^2dxdt+s^3\iint_{\omega_0\times(0,T)}e^{-2s\alpha}\xi^3|u|^2dxdt\right)
\end{split}
\end{equation}
%
for every $s\geq C$. 
\end{lemma}
%
\begin{remark}
In \cite{fursi,cara_guerrero}, the authors use the function $l(t)=t(T-t)$ to prove the above lemma. Using the weights in\eqref{weights_l} does not change the result, since the important property is that $l(t)$ goes to 0 as $t$ tends to 0 a $T$.
\end{remark}

Using the previous results, we can prove a Carleman inequality for the solutions of system \eqref{adj_sys_foll}. This will be the main ingredient to prove the observability inequality \eqref{obs_ineq_1}. The result is the following:
%
\begin{proposition}\label{prop_carleman}
Under assumptions of Theorem \ref{main_1}. Then, there exists a constant $C$ such that for any $\varphi^T\in H^1_0(\Omega)$, the solution $(\varphi,\theta)$ to \eqref{adj_sys_foll} satisfies
%
\begin{equation}\begin{split}
I(s;\varphi)+\widetilde I(s;\theta) \leq C\left(s^5\iint_{\omega\times(0,T)}e^{-2s\alpha}\xi^5|\varphi|^2\right)
\end{split}
\end{equation}
%
for every $s>0$ large enough.
%
\end{proposition}
%
\begin{proof}
We start by applying inequality \eqref{car_clasica} to the first equation in \eqref{adj_sys_foll} and inequality \eqref{car_boundary} to the second one. We take $\hat \lambda=\max\{\lambda_0,\lambda_1\}$ and fix $\lambda\geq \hat \lambda$. Adding them up, we obtain
%
\begin{equation}
\begin{split}
I(s;\varphi)+\widetilde{I}(s;\theta)\leq C&\left(\iint_{\omega_0\times(0,T)}e^{-2s\alpha}\left(s^3\xi^3|\varphi|^2+s\xi|\theta|^2\right) +\iint_Q e^{-2s\alpha}\left(|\theta\mathbf{1}_{\mathcal O_d}|^2+s^{-2}\xi^{-2}|\tfrac{1}{\gamma^2}\varphi|^2\right)\right. \\
&\quad\left. +s^{-\frac{1}{2}}\left\|e^{-s\alpha}\xi^{-\frac{1}{4}}\frac{1}{\ell^2}\frac{\partial \varphi}{\partial \nu}\right\|^2_{H^{\frac{1}{4},\frac{1}{2}}(\Sigma)}+s^{-\frac{1}{2}}\left\| e^{-2s\alpha}\xi^{-\frac{1}{4}+\frac{1}{m}} \frac{1}{\ell^2}\frac{\partial \varphi}{\partial \nu} \right\|^{2}_{L^2(\Sigma)}  \right)
\end{split}
\end{equation}
%
Taking the parameters $s$ large enough, we can absorb the lower order terms into the left-hand side in the previous inequality. More precisely, we have 
%
\begin{equation}\label{car_con_bound}
\begin{split}
I(s;\varphi)+\widetilde{I}(s;\theta)\leq C&\left(\iint_{\omega_0\times(0,T)}e^{-2s\alpha}\left(s^3\xi^3|\varphi|^2+s\xi|\theta|^2\right) + s^{-\frac{1}{2}}\left\| e^{-2s\alpha}\xi^{-\frac{1}{4}+\frac{1}{m}} \frac{1}{\ell^2}\frac{\partial \varphi}{\partial \nu} \right\|^{2}_{L^2(\Sigma)} \right. \\
&\quad\left. +s^{-\frac{1}{2}}\left\|e^{-s\alpha}\xi^{-\frac{1}{4}}\frac{1}{\ell^2}\frac{\partial \varphi}{\partial \nu}\right\|^2_{H^{\frac{1}{4},\frac{1}{2}}(\Sigma)}  \right)
\end{split}
\end{equation}
%
for all $s\geq C$, with $C>0$ only depending on $\Omega$, $\omega$, $\mathcal O_d$ and $\lambda$. 

Then, we proceed to estimate the first boundary term in \eqref{car_con_bound}. Reasoning as in \eqref{duprez_lissy}, we take a functon $\kappa\in C^2(\overline \Omega)$ such that
%
\begin{equation}
\frac{\partial \kappa}{\partial \nu}=1 \quad\text{and}\quad \kappa=1 \quad \text{on } \Gamma.
\end{equation}

Note that from the definition of the weight functions \eqref{weights_l}, we have $\alpha$ and $\alpha^*$ are equal in $\Gamma$, thus we may write
%
\begin{equation}
\begin{split}
\iint_{\Sigma}e^{-2s\alpha}\left|\frac{\partial \varphi}{\partial \nu}\right|^2&=\int_{0}^{T}e^{-2s\alpha^*}\int_\Gamma\nabla\varphi\cdot \nabla \kappa\frac{\partial \varphi}{\partial \nu} \\
&=\int_{0}^{T}e^{-2s\alpha^*}\int_{\Omega}\Delta \varphi\,(\nabla\varphi\cdot \nabla \kappa)+\int_{\Omega}e^{-2s\alpha^*}\int_\Omega\nabla (\nabla\varphi\cdot\nabla\kappa)\cdot\nabla \varphi
\end{split}
\end{equation}
%
where we have integrated by parts in the right-hand side. Using Cauchy-Schwarz and Young inequalities, we get
%
\begin{equation}\label{estimate_1}
\begin{split}
\iint_{\Sigma}e^{-2s\alpha}\left|\frac{\partial \varphi}{\partial \nu}\right|^2&\leq C\left(\iint_Q e^{-2s\alpha^*}\left((s\xi)^{-1}|\Delta \varphi|^2 + s\xi|\nabla \varphi|^2\right)+\int_{0}^{T}e^{-2s\alpha^*}(s\xi^*)^{-1}\|\varphi\|^2_{H^2(\Omega)}\right).
\end{split}
\end{equation}

We proceed to estimate the last term in the above inequality. Let us set $\widehat \varphi=\sigma\varphi$ with $\sigma\in C^\infty([0,T])$ defined as
%
\begin{equation}\label{rho_def}
\sigma:=e^{-s\alpha^*}(s\xi^*)^a
\end{equation}
%
for some $a\in \mathbb R$ to be chosen later. Observe that $\rho(T)=0$. Then, $\widehat{\varphi}$ is solution to the system
%
\begin{equation}\label{phi_gorro}
\begin{cases}
-\widehat\varphi_t-\Delta \widehat\varphi=\sigma\,\theta\mathbf{1}_{\mathcal O_d}-\sigma_t\varphi \quad& \text{in } Q, \\
\widehat\varphi=0 \quad& \text{on } \Sigma, \\
\widehat{\varphi}(\cdot,T)=0 \quad&\text{in } \Omega.
\end{cases}
\end{equation}
 
From standard regularity estimates for the heat equation, we have that $\widehat{\varphi}$ solution to \eqref{phi_gorro} satisfies
%
\begin{equation}\label{est_regul}
\|\widehat{\varphi}\|_{L^2(0,T;H^2(\Omega))\cap H^1(0,T;L^2(\Omega))}\leq C\left(\|\sigma\,\theta\|_{L^2(Q)}+\|\sigma_t\varphi\|_{L^2(Q)}\right).
\end{equation}
%
Using the definitions of $\alpha^*$ and $\xi^*$ given in \eqref{weights_l} together with \eqref{rho_def}, it is not difficult to see that 
%
\begin{equation}
|\sigma_t|\leq Ce^{-s\alpha^*}(s\xi^*)^{a+{(m+1)}/{m}}.
\end{equation}
%
Using \eqref{est_regul} with $\sigma=e^{-s\alpha^*}(s\xi^*)^{-1/2}$, we obtain
%
\begin{equation}\label{estimate_2}
\int_{0}^T e^{-2s\alpha^*}(s\xi^{*})^{-1}\|\varphi\|_{H^2(\Omega)}^2\leq C\left(\iint_Qe^{-2s\alpha^*}s\xi^*|\theta|^2+\iint_Qe^{-2s\alpha^*}(s\xi^*)^{-1+2(m+1)/m}|\varphi|^2\right)
\end{equation}

Putting together estimates \eqref{estimate_1} and \eqref{estimate_2}, and using the definitions of the weights $\alpha^*$ and $\xi^*$, we obtain
%
\begin{equation*}
\begin{split}
\iint_{\Sigma}e^{-2s\alpha}\left|\frac{\partial \varphi}{\partial \nu}\right|^2&\leq C\left(\iint_Q e^{-2s\alpha}\left((s\xi)^{-1}|\Delta \varphi|^2 + s\xi|\nabla \varphi|^2\right)+\iint_Qe^{-2s\alpha}\left(s\xi|\theta|^2+(s\xi)^{3/2}|\varphi|^2\right)\right).
\end{split}
\end{equation*}
%
Thus, $\|e^{-s\alpha}\frac{\partial \varphi}{\partial \nu}\|_{L^2(\Sigma)}^2$ is bounded by the left-hand side of \eqref{car_con_bound}. By taking $s$ large enough, we can now absorb the boundary term 
%
\[
s^{-\frac{1}{2}}\left\| e^{-2s\alpha}\xi^{-\frac{1}{4}+\frac{1}{m}} \frac{1}{\ell^2}\frac{\partial \varphi}{\partial \nu} \right\|^{2}_{L^2(\Sigma)}.
\]

To treat the second boundary term in \eqref{car_con_bound}, we use the trace estimate in Lemma \ref{ap_1}. More precisely, we have 
%
\begin{equation}\label{est_trace}
\begin{split}
s^{-\frac{1}{2}}\left\|e^{-s\alpha}\xi^{-\frac{1}{4}} \frac{\partial \varphi}{\partial \nu} \right\|^2_{H^{\frac{1}{2},\frac{1}{4}}(\Sigma)}&=s^{-\frac{1}{2}}\left\|e^{-s\alpha^*}(\xi^*)^{-\frac{1}{4}} \frac{\partial \varphi}{\partial \nu} \right\|^2_{H^{\frac{1}{2},\frac{1}{4}}(\Sigma)} \\
&\leq s^{-\frac{1}{2}}\left\| e^{-s\alpha^*}(\xi^*)^{-\frac{1}{4}} \varphi \right\|_{H^{2,1}(Q)}
\end{split}
\end{equation}

Now, set $\widehat \varphi=\sigma \varphi$ with $\sigma=e^{-s\alpha^*}(s\xi)^{-\frac{1}{4}}$. Arguing as before, we readily obtain from \eqref{est_regul} the following 
%
\begin{equation}\label{est_frontera}
s^{-\frac{1}{2}}\left\| e^{-s\alpha^*}(\xi^*)^{-\frac{1}{4}} \varphi \right\|_{H^{2,1}(Q)}\leq C\left(\iint_{Q}e^{-2s\alpha^*}(s\xi^*)^{-\frac{1}{2}}|\theta|^2+\iint_{Q}e^{-2s\alpha^*}(s\xi^*)^2|\varphi|^2\right).
\end{equation}
%
Putting together \eqref{est_trace} and \eqref{est_frontera} and substituting in \eqref{car_con_bound}, we can take $s$ sufficiently large to absorb the remaining terms. 

Up to now, we have 
%
\begin{equation}\label{est_locales}
\begin{split}
I(s;\varphi)+\widetilde{I}(s;\theta)\leq C&\left(\iint_{\omega_0\times(0,T)}e^{-2s\alpha}\left(s^3\xi^3|\varphi|^2+s\xi|\theta|^2\right) \right)
\end{split}
\end{equation}
%
for all $s$ large enough. The last step is to eliminate the local term corresponding to $\theta$. To this end, consider an open set $\omega_1$ such that $\omega_0\subset\subset\omega_1\subset \subset \omega\cap \mathcal O_d$ and a function $\zeta\in C_0^2(\omega_1)$ verifying 
%
\begin{equation}\label{cut_off}
\zeta\equiv 1\quad \text{in } \omega_0 \quad \text{and} \quad 0\leq \zeta\leq 1. 
\end{equation}
%
Thanks to hypothesis \eqref{cond_omega}, we use the first equation of \eqref{adj_sys_foll} and \eqref{cut_off} to obtain
%
\begin{equation}
\iint_{\omega_0\times(0,T)}e^{-2s\alpha}s\xi|\theta|^2\leq \iint_{\omega_1\times(0,T)}e^{-2s\alpha}s\xi \theta(-\varphi_t-\Delta \varphi)\zeta
\end{equation}
%
Arguing as in \cite{luz_manuel,deteresa2000,b_gb_r_2}, we integrate by parts in the right-hand side of the above inequality. Then, using the second equation of \eqref{adj_sys_foll}, \eqref{cut_off} and the definition of the weight functions \eqref{weights_l}, it is not difficult to see that 
%
\begin{equation}
\begin{split}
\iint_{\omega_0\times(0,T)}e^{-2s\alpha}s\xi|\theta|^2\leq& \frac{1}{\gamma^2}\iint_{Q}e^{-2s\alpha}s\xi |\varphi|^2+\iint_{\omega_1\times(0,T)}e^{-2s\alpha}\left(s^3\xi^3|\theta| |\varphi|+s^2\xi^2|\nabla \theta||\varphi|\right).
\end{split}
\end{equation}
%
Using H\"older and Young inequalities, it follows 
%
\begin{equation}\label{est_final}
\begin{split}
\iint_{\omega_0\times(0,T)}e^{-2s\alpha}s\xi|\theta|^2\leq& \frac{1}{\gamma^2}\iint_{Q}e^{-2s\alpha}s\xi |\varphi|^2+\varepsilon \widetilde I(s;\theta)+\iint_{\omega_1\times(0,T)}e^{-2s\alpha}s^5\xi^5|\varphi|^2.
\end{split}
\end{equation}
%
We replace \eqref{est_final} in \eqref{est_locales} with $\varepsilon$ small enough and  since $\gamma$ is sufficiently large, we can absorb the first two terms of the above inequality. Finally, noting that $\omega_1\subset \omega$, we obtain the desired inequality. This concludes the proof of Proposition \ref{prop_carleman}
%
\end{proof}

\subsubsection*{The observability inequality}


\magenta{Modificando desde ac\'a}



\section{The case with distributed follower and boundary leader}

Now we will considere systems of the form 
%
\begin{equation}\label{heat_dif1}
\begin{cases}
z_t-\Delta z=\csin{\mathcal B_1}v+\csin{\mathcal B_2}\psi, & \text{in Q}, \\
z=h\csbd &\text{on } \Sigma, \\
z(x,0)=z^0(x), & \text{in } \Omega.
\end{cases}
\end{equation}

The proof of the existence and uniqueness of the saddle point of \eqref{heat_dif1} is equivalent to the problem \magenta{(\ref{}introduzca la cita adecuada)}. In consequence, we get the following coupled system 
%
\begin{equation}\label{couple1}
\begin{cases}
z_t-\Delta z=-\frac{1}{\ell^2}p\csin{\mathcal B_1}+\frac{1}{\gamma^2}p, & \text{in Q}, \\
-p_t-\Delta p=(z-z_d)\csin{\mathcal \cbd_d}, & \text{in Q}, \\
z=h\csbd, \quad p=0 &\text{on } \Sigma, \\
z(x,0)=z^0(x), \quad p(x,T)=p^T(x)& \text{in } \Omega.
\end{cases}
\end{equation}

To find the optimal control $\bar h$ such that minimizes the functional \eqref{opt_leader} we will use the Fenchel-Rockafellar technique, hence, it is necessary consider the adjoint system of \eqref{couple1} 

\begin{equation}\label{adjunto1}
\begin{cases}
-\varphi_t-\Delta \varphi=\theta\csin{\mathcal \cbd_d}, & \text{in Q}, \\
\theta_t-\Delta \theta=-\frac{1}{\ell^2}\varphi\csin{\mathcal B_1}+\frac{1}{\gamma^2}\varphi, & \text{in Q}, \\
\theta=0, \quad \varphi=0 &\text{on } \Sigma, \\
\theta(x,0)=0, \quad \varphi(x,T)=\varphi^T(x)& \text{in } \Omega,
\end{cases}
\end{equation}
%
and  define the dual functional 

\begin{equation}\label{func_FR}
F_{\epsilon}(\varphi^T)=\frac{1}{2}\iint_{\Sigma}\left|\frac{\partial \varphi}{\partial n}\csbd \right|^2\dx\dt+\frac{\epsilon}{2}\left\|\varphi^T\right\|+\int_{\Omega}\varphi(0)z^0(x)\dx-\iint_{Q}\theta z_d \dx\dt
\end{equation}

Previously to proof our first result it is necessary intoduced the following notation. 

\begin{notation}\magenta{creo que no es notaci\'on. Por otro lado, no s\'e quien es $S^{'}$, creo que $S^{\prime}\subset\subset \mathcal{O}$}
\begin{equation*}
\begin{split}
&\beta_i=\begin{cases}
     \frac{e^{\lambda(\|\eta_1\|+\|\eta_2\|)}-e^{\lambda \eta_i(x)}}{T^4/16}, & \text{on } [0,T/2]\\
     \tilde{\alpha}_i & \text{on } [T/2,T],
\end{cases}\\
&\phi_i=\begin{cases}
     \frac{e^{\lambda \eta_i(x)}}{T^4/16}, & \text{on } [0,T/2]\\
      \tilde{\xi}_i & \text{on } [T/2,T],
\end{cases}\\
\end{split}\end{equation*}
%
where
%
\begin{equation*}
\begin{split}
\tilde{\xi}_i=&\frac{e^{\lambda\eta_i(x)}}{t^2(T-t)^2},\qquad \tilde{\alpha}_i=\frac{e^{\lambda(\|\eta_1\|+\|\eta_2\|)}-e^{\lambda \eta_i(x)}}{t^2(T-t)^2}.
\end{split}\end{equation*}
Set $\bar{\eta} \in C^1([0,T])$ such that 
\begin{equation}
\bar{\eta}=\begin{cases}
    1,  & \text{on }[0,T/2] \\
    0, & \text{on } [3T/4,T].
\end{cases}\end{equation}
\end{notation}
%
\begin{remark}
We can note that,
\begin{equation*}
\tilde{\xi}_2\leq \tilde{\xi}_1,\quad \alpha_1\leq \alpha_2\quad\text{in}\quad \Omega\quad\text{and}\quad\tilde{\xi}_2= \tilde{\xi}_1,\quad \alpha_1=\alpha_2\quad\text{in}\quad \mathcal{O}_d. 
\end{equation*}
\end{remark} 

\begin{proposition}
\magenta{Under the hypotheses .... $\Sigma^{\prime}=S^{\prime}\times (0,T)$}. Then, there exists a constant $C$ \magenta{maybe $\sigma$, $s$ large enough} such that
%

\begin{equation}
\left\|\varphi(x,0)\right\|^2_{L^2(\Omega)}+\iint_{Q}\sigma^2|\theta|^2dxdt\leq C\iint_{\Sigma}\left|\frac{\partial \varphi}{\partial n}\csbd \right|^2\dx\dt
\end{equation}
 \end{proposition}
 
 \begin{proof}
 We begin with the following Carleman estimate \magenta{\cite{} fernandez cara},
 
\magenta{ \begin{equation*}
\iint_{Q}e^{-2s\tilde{\alpha}}s\tilde{\xi}|\nabla q|^2\dx\dt+\iint_{Q}e^{-2s\tilde{\alpha}}s^2\lambda^2\tilde{\xi}^3|q|^2\leq C\left(\iint_{Q}e^{-2s\tilde{\alpha}}|f|^2+s\iint_{\Sigma}e^{-2s\tilde{\alpha}}\tilde{\xi}\left|\frac{\partial q}{\partial n}\right|^2\dx\dt\right),
\end{equation*}}
%
\magenta{for $\lambda>1$. Hubo un procedimiento para llegar al siguiente estimado usando el anterior. No tengo le procedimiento y no lo repet\'i, adem\'as no s\'e que pas\'o con la $\lambda^2$} 
 \begin{equation}\label{Est1}
 \begin{split}
\iint_{Q}e^{-2s\tilde{\alpha}_i}(s\tilde{\xi}_i)^{1+m}|\nabla q|^2\dx\dt+&\iint_{Q}e^{-2s\tilde{\alpha}_i}(s\tilde{\xi}_i)^{3+m}|q|^2\dx\dt\\
&\leq C\left(\iint_{Q}e^{-2s\tilde{\alpha}_i}(s\tilde{\xi}_i)^m|f|^2\dx\dt+\iint_{\Sigma^{\prime}}e^{-2s\tilde{\alpha}_i}(s\tilde{\xi}_i)^{1+m}\left|\frac{\partial q}{\partial n}\right|^2\dx\dt\right),
\end{split}
\end{equation}
%
for $m\in\mathbb{R}$ and $i=1,2$.

Applying the estimate \eqref{Est1} to the couple system \eqref{adjunto1} implies 

\begin{align}\notag
&I(\varphi,1,0)+I(\theta, 2, -1)\\\notag
&\leq C\left(\iint_{Q}e^{-2s\tilde{\alpha}_1}|\theta|^2\dx\dt+\iint_{\Sigma^{\prime}}e^{-2s\tilde{\alpha}_1}(s\tilde{\xi}_1)\left|\frac{\partial \varphi}{\partial n}\right|^2\dx\dt\right.\\ \notag
&\quad+\left.\iint_{Q}e^{-2s\tilde{\alpha}_2}(s\tilde{\xi}_2)^{-1}|-\frac{1}{\ell^2}\varphi\csin{\mathcal B_1}+\frac{1}{\gamma^2}\varphi|^2\dx\dt+\iint_{\Sigma^{\prime}}e^{-2s\tilde{\alpha}_2}\left|\frac{\partial \theta}{\partial n}\right|^2\dx\dt\right)\\ \label{igualda1}
&\leq C\left(\iint_{\Sigma^{\prime}}e^{-2s\tilde{\alpha}_1}(s\tilde{\xi}_1)\left|\frac{\partial \varphi}{\partial n}\right|^2\dx\dt+\iint_{\Sigma^{\prime}}e^{-2s\tilde{\alpha}_2}\left|\frac{\partial \theta}{\partial n}\right|^2\dx\dt\right).
\end{align}
%
where
\begin{equation}\label{nota1}
I(\cdot,i,m):=\iint_{Q}e^{-2s\tilde{\alpha}_i}(s\tilde{\xi}_i)^{1+m}|\nabla \cdot|^2\dx\dt+\iint_{Q}e^{-2s\tilde{\alpha}_i}(s\tilde{\xi}_i)^{3+m}|\cdot|^2\dx\dt.
\end{equation}
Note that we can replace the functions $\tilde{\xi}_i, \tilde{\alpha}_i$ by $\beta_i, \phi_i$ in the interval $[T/2,T]$ in the second integral of \eqref{nota1} hence,

\begin{equation}
\int_{T/2}^{T}\int_{\Omega}e^{-2s\beta_i}(s\phi_i)^{3+m}|\cdot|^2\dx\dt\leq I(\cdot,i,m).
\end{equation}
%
In consequence and from inequality \eqref{igualda1} we get


\begin{align}\notag
&\int_{T/2}^{T}\int_{\Omega}e^{-2s\beta_1}\phi_1^{3}|\varphi|^2\dx\dt+\int_{T/2}^{T}\int_{\Omega}e^{-2s\beta_2}(s\phi_2)^{2}|\theta|^2\dx\dt\\
&\leq C\left(\iint_{\Sigma^{\prime}}e^{-2s\beta_1}\phi_1\left|\frac{\partial \varphi}{\partial n}\right|^2\dx\dt+\iint_{\Sigma^{\prime}}e^{-2s\beta_2}\left|\frac{\partial \theta}{\partial n}\right|^2\dx\dt\right),
\end{align}
%
for some constant $C=C(s,T)$.
\

\magenta{viene lo de las hojas que me mandaste}
\

\end{proof}

\frenchspacing
\begin{thebibliography}{7}

\bibitem{a_araujo}
\textsc{F. D. Araruna, B. S. V. Ara\'ujo and E. Fern\'andez-Cara.}
\newblock Stackelberg-Nash null controllability for some linear and semilinear degenerate parabolic equations. 
\newblock {\em Math. Control Signals Systems}, \textbf{30} (2018).

\bibitem{araruna}
\textsc{F. D. Araruna, E. Fern\'andez-Cara, and M. C. Santos.}
\newblock {S}tackelberg-{N}ash exact controllability for linear and semilinear parabolic equations.
\newblock {\em ESAIM: Control Optim. Calc. Var.}, \textbf{21}, 3 (2015), 835--856.
%
\bibitem{araruna1}
\textsc{F.D. Araruna, E. Fern\'andez-Cara, S. Guerrero, \& M. C.  Santos.} 
\newblock New results on the Stackelberg Nash exact controllability for parabolic equations. 
\newblock {\em Systems \& Control Letters}, \textit{104} (2017)., 78--85. 
%
\bibitem{da_silva}
\textsc{F. D. Araruna, E. Fern\'andez-Cara and L. C. da Silva}
\newblock Hierarchical exact controllability of semilinear parabolic equations with distributed and boundary controls.
\newblock {\em Preprint}, (2018). 
%
\bibitem{AMR}
\textsc{F.D. Araruna,  S.D.B. de Menezes, and  M.A. Rojas-Medar.}
\newblock On the approximate controllability of Stackelberg-Nash strategies for linearized microplar fluids.
\newblock {\em Applied Mathematics \& Optimization},  \textbf{70},  3 (2014), 373--393.
%
\bibitem{assia_survey}
\textsc{F. Ammar-Khodja, A. Benabdallah, M. Gonz\'alez-Burgos, and L. de Teresa.}
\newblock Recent results on the controllability of linear coupled parabolic problems: a survey. 
\newblock {\em Math. Control Relat. Fields}, \textbf{1} (2011), no. 3, 267--306.
%
\bibitem{assia_luz_new}
\textsc{F. Ammar-Khojda, A. Benabdallah, M. Gonz\'alez-Burgos, and L. de Teresa.}
\newblock New phenomena for the null controllability of parabolic systems: Minimal time and geometrical dependence.
\newblock {\em J. Math. Anal. Appl.}, \textbf{444}, 2 (2016), 1071--1113.

%
\bibitem{aziz}
\textsc{A. Belmiloudi.}
\newblock On some robust control problems for nonlinear parabolic equations.
\newblock {\em Int. J. Pure Appl. Math.}, \textbf{11}, 2 (2004), 119--149.
%
\bibitem{temam}
\textsc{T. R. Bewley, R. Temam, and M. Ziane.}
\newblock A generalized framework for robust control in fluid mechanics. 
\newblock {\em Center for Turbulence Research Annual Briefs}, (1997), 299--316.
%
\bibitem{temam_nonlinear}
\textsc{T. R. Bewley, R. Temam, and M. Ziane.}
\newblock A general framework for robust control in fluid mechanics.
\newblock {\em Physica D}, \textbf{138} (2000), 360--392.
%
\bibitem{b_gb_r_2}
\textsc{O. Bodart, M. Gonz\'alez-Burgos, and R. P\'erez-Garc\'ia.}
\newblock Insensitizing controls for a heat equation with a nonlinear term involving the state and the gradient. 
\newblock {\em Nonlinear Anal.}, \textbf{57}, 5-6 (2004), 687--711.
%
\bibitem{carreno}
\textsc{N. Carre\~{n}o and M. C. Santos}.
\newblock Stackelberg-Nash exact controllability for the Kuramoto-Sivashinsky equation.
\newblock {\em Preprint}, (2018). 
%
\bibitem{Ekeland}
\textsc{I. Ekeland and R. Temam.}
\newblock Convex analysis and variational problems.
\newblock {\em North-Holland}, (1976).
%
%\bibitem{evans}
%\textsc{L. C. Evans.}
%\newblock Partial differential equations. 
%\newblock {\em Graduate studies in Mathematics, AMS}, Providence, (1991). 
%%
%\bibitem{zuazua_fer}
%\textsc{L. A. Fern\'andez and E. Zuazua.}
%\newblock Approximate controllability for the semilinear heat equation involving gradient terms.
%\newblock {\em J. Optim. Theor. Appl.,} \textbf{101}, 2 (1999), 307--328.
%
%\bibitem{zuazua_fabre}
%\textsc{C. Fabre, J. P. Puel, and E. Zuazua.}
%\newblock Approximate controllability of the semilinear heat equation.
%\newblock {\em Proc. Roy. Soc. Edinburgh Sect. A} \textbf{125}, 3 (1995), 31--61.
%
\bibitem{cara_guerrero}
\textsc{E. Fern\'andez-Cara and S. Guerrero.}
\newblock Global {C}arleman inequalities for parabolic systems and applications to controllability.
\newblock {\em SIAM J. Control Optim.}, \textbf{45}, 4 (2006), 1395--1446.
%
%\bibitem{cara_NS}
%\textsc{E. Fern\'andez-Cara, S. Guerrero, O. Yu. Imanuvilov and  J. P.  Puel.}  Local exact controllability of the Navier-Stokes system. {\em J. Math. Pures Appl.} \textbf{83}, 12 (2004),  1501--1542.
%%
%\bibitem{fc_zuazua}
%\textsc{E. Fern\'andez-Cara and E. Zuazua.}
%\newblock Null and approximate controllability for weakly blowing up semilinear heat equations. 
%\newblock {\em Ann. I. H. Poincar\'e-AN}, \textbf{17}, 5 (2000), 583--616.
%
\bibitem{fursi}
\textsc{A. Fursikov and O. Yu. Imanuvilov.}
\newblock Controllability of evolution equations.
\newblock {\em Lecture Notes, Research Institute of Mathematics}, Seoul National University, Korea, (1996).
%
\bibitem{Glowinski}
\textsc{R. Glowinski, A. Ramos, and J. Periaux.}
\newblock Nash equilibria for the multiobjective control of linear partial
  differential equations.
\newblock {\em J. Optim. Theory Appl.}, \textbf{112}, 3 (2002), 457--498.
%
\bibitem{luz_manuel}
\textsc{M. Gonz\'alez-Burgos and L. de Teresa.}
\newblock Controllability results for cascade systems of $m$ coupled parabolic PDEs by one control force.
\newblock {\em Portugaliae Mathematica,} \textbf{67}, 1 (2010), 91--113.
%
\bibitem{Guillen}
\textsc{F. Guill\'en-Gonz\'alez, F. Marques-Lopes, and M. Rojas-Medar.}
\newblock On he approximate controllability of {S}tackelberg-{N}ash strategies
  for {S}tokes equations.
\newblock {\em Proceedings of the American Mathematical Society}, \textbf{141}, 5 (2013),
  1759--1773.
 %
\bibitem{vhs_deT_rob}
\textsc{V. Hern\'andez-Santamar\'ia and L. de Teresa}
\newblock Robust Stackelberg controllability for linear and semilinear heat equations.
\newblock {\em Preprint}, (2017).
 % 
%\bibitem{ima_yama}
%\textsc{O. Yu. Imanuvilov and M. Yamamoto.}
%\newblock Carleman inequalities for parabolic equations in {S}obolev spaces of negative order and exact controllability for semilinear parabolic equations.
%\newblock {\em Publ. RIMS, Kyoto Univ.}, \textbf{39}, (2003), 227--274.
%%
% \bibitem{lady}
%\textsc{O. A. Ladyzhenskaya, V. A. Solonnikov, and N. N. Ural'ceva.}
%\newblock Linear and quasi-linear equations of parabolic type.
%\newblock {\em Translations of Mathematical Monographs 23} (1968).
%
\bibitem{Limaco}
\textsc{J. Limaco, H. Clark, and L. Medeiros.}
\newblock Remarks on hierarchic control.
\newblock {\em Journal of Mathematical Analysis and Applications}, \textbf{359}, 1
  (2009), 368--383.
%
%\bibitem{Lions_optim}
%\textsc{J.-L Lions.}
%\newblock Optimal control of systems governed by partial differential equations. 
%\newblock {\em Springer-Verlag}, 1971.
%
\bibitem{LionsHier}
\textsc{J.-L. Lions.}
\newblock Hierarchic control.
\newblock {\em Proceedings of the Indian Academy of Science (Mathematical
  Sciences),} \textbf{104}, 1 (1994), 295--304.
%
\bibitem{LionsSta}
\textsc{J.-L. Lions.}
\newblock Some remarks on {S}tackelberg's optimization.
\newblock {\em Mathematical Models and Methods in Applied Sciences 4}, 4
  (1994), 477--487.
%
%\bibitem{Nash}
%\textsc{J. F. Nash.}
%\newblock Non-cooperative games.
%\newblock {\em Annals of Mathematics,} \textbf 54, 2 (1951), 286--295.
%%
%\bibitem{Pareto}
%\textsc{V. Pareto.}
%\newblock Cours d'\'economie politique.
%\newblock {\em Switzerland\/} (1896).
%%
%\bibitem{seidman}
%\textsc{T. Seidman and H.Z. Zhou.}
%\newblock Existence and uniqueness of optimal controls for a quasilinear parabolic equation.
%\newblock {\em SIAM J. Control Optim.}, \textbf{20}, 6 (1982), 747--762.
%
\bibitem{Stackelber}
\textsc{H. von Stackelberg.}
\newblock Marktform und {G}leichgewicht.
\newblock {\em Springer\/} (1934).
%
\bibitem{deteresa2000}
\textsc{L. de Teresa.}
\newblock Insensitizing controls for a semilinear heat equation.
\newblock {\em Comm. Partial Differential Equations,} \textbf{25}, 1--2 (2000) 39--72.
%
%\bibitem{trol}
%\textsc{F. Tr\"oltzsch.}
%\newblock Optimal control of partial differential equations: theory, methods and applications.
%\newblock {\em American Mathematical Society}, (2010).
%
%\bibitem{Ekeland}
%{I. Ekeland and R. Temam.}
%\newblock Convex analysis and variational problems.
%\newblock North-Holland, 1976.
%
%
%
%\bibitem{zuazua_manuel}
%\textsc{Doubova, A., Fern\'andez-Cara, E., Gonz\'alez-Burgos, M., and Zuazua, E.}
%\newblock On the controllability of parabolic systems with a nonlinear term involving the state and the gradient.
%\newblock {\em SIAM Journal of Control and Optimization 41}, 3 (2002), 798--819.

\bibitem{Ekeland}
\textsc{I. Ekeland and R. Temam.}
\newblock Convex analysis and variational problems.
\newblock {\em North-Holland}, (1976).
%
%
%
%
%\bibitem{fabre}
%\textsc{C. Fabre, J.P. Puel, and E. Zuazua.}
%\newblock Approximate controllability of the semilinear heat equation. 
%\newblock {\em Proc. Royal Soc. Edinburgh}, \textbf{125} A, (1995), 31--61.
%%
%%
%\bibitem{fc_zuazua}
%\textsc{E. Fern\'andez-Cara and E. Zuazua.}
%\newblock Null and approximate controllability for weakly blowing up semilinear heat equations. 
%\newblock {\em Ann. I. H. Poincar\'e-AN}, \textbf{17}, 5 (2000), 583--616.
%

%\bibitem{zuazua_cara}
%\textsc{Fern\'andez-Cara, E., and Zuazua, E.}
%\newblock The cost of approximate controllability for heat equations: the linear case.
%\newblock {\em Adv. Differential Equations 5}, 4--6 (2000), 465--514.


\bibitem{fursi}
\textsc{A. Fursikov and O. Yu. Imanuvilov.}
\newblock Controllability of evolution equations.
\newblock {\em Lecture Notes, Research Institute of Mathematics}, Seoul National University, Korea, (1996).

%\bibitem{glo_lions}
%\textsc{R. Glowinski and J.-L. Lions.}
%\newblock Exact and approximate controllability for distributed parameter systems.
%\newblock {\em Acta Numer.,} (1994), 269--378.
%
%\bibitem{glo_lions_he}
%\textsc{R. Glowinski, J.-L. Lions, and J. He.}
%\newblock Exact and approximate controllability for distributed parameter systems.
%\newblock {\em Encyclopedia of Mathematics and its Applications}, vol. 117, Cambridge University Press, Cambridge, (2008).
%
\bibitem{luz_manuel}
\textsc{M. Gonz\'alez-Burgos and L. de Teresa.}
\newblock Controllability results for cascade systems of $m$ coupled parabolic PDEs by one control force.
\newblock {\em Portugal. Math.}, \textbf{67}, 1 (2010), 91--113.

%\bibitem{guerrero_stokes}
%\textsc{S. Guerrero.}
%\newblock Controllability of systems of Stokes equations with one control force: existence of insensitizing controls.
%\newblock {\em Ann. I. H. Poincar\'e--AN}, \textbf{24}, 6 (2007), 1029--1054.
%
%\bibitem{guerrero}
%\textsc{S. Guerrero.}
%\newblock Null controllability of some systems of two parabolic equations with one control force.
%\newblock {\em SIAM J. Control Optim.} \textbf{46}, 2 (2007), 379--394.
%%
%\bibitem{mamadou}
%\textsc{M. Gueye.}
%\newblock Insensitizing controls for the Navier-Stokes equations.
%\newblock {\em Ann. I. H. Poincar\'e--AN}, \textbf{30}, 5 (2013), 825--844.
%
\bibitem{jesus}
\textsc{I. P. de Jesus, J. Limaco and M. R. Clark.}
\newblock Hierarchical control for the one-dimensional plate equation with a moving boundary. 
\newblock {\em J. Dyn. Control Syst.}, \textbf{24} (2018), no. 4, 635--655.
%
%
\bibitem{montoya}
\textsc{C. Montoya and L. de Teresa.}
\newblock Robust Stackelberg controllability for the Navier-Stokes equations. 
\newblock {\em NoDEA Nonlinear Differential Equations Appl.}, \textbf{25} (2018).
%
%\bibitem{munch_zuazua}
%\textsc{A. M\"unch and E. Zuazua.}
%\newblock Numerical approximation of null controls for the heat equation: ill-posedness  and remedies. 
%\newblock {\em Inverse problems,} \textbf{26}, 8 (2010), 085018, 39pp.
%  
% \bibitem{mauffrey}
% \textsc{Mauffrey, K.}
% \newblock Contr\^olabilit\'e de syst\`emes gouvern\'es par des \'equations aux d\'eriv\'ees partielles. 
% Analysis of PDEs. Universit\'e de Franche-Comt\'e. French. 
%
\bibitem{deteresa2000}
\textsc{L. de Teresa.}
\newblock Insensitizing controls for a semilinear heat equation.
\newblock {\em Comm. Partial Differential Equations}, \textbf{25}, 1--2 (2000) 39--72.

%\bibitem{deteresa_zuazua}
%\textsc{L. de Teresa and E. Zuazua.}
%\newblock Identification of the class of initial data for the insensitizing control of the heat equation.
%\newblock {\em Commun. Pure. Appl. Anal.,} \textbf{8}, 1 (2009) 457--471.
%%
%%
%
%\bibitem{zuazua_num}
%\textsc{E. Zuazua.}
%\newblock Control and numerical approximation of the wave and heat equation.
%\newblock {\em International Congress of Mathematicians,} Madrid, Spain \textbf{III} (2006) 1389--1417.


\end{thebibliography}

\end{document}
